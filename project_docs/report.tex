\documentclass{article}

% ICLR Paper Style
\usepackage[preprint]{neurips_2024}
\usepackage[utf8]{inputenc}
\usepackage[T1]{fontenc}
\usepackage{hyperref}
\usepackage{url}
\usepackage{booktabs}
\usepackage{amsfonts}
\usepackage{nicefrac}
\usepackage{microtype}
\usepackage{graphicx}
\usepackage{subcaption}
\usepackage{amsmath}
\usepackage{algorithm}
\usepackage{algorithmic}

% Title and Author Information
\title{SimRAG Reproduction: A Simplified Implementation of\\Retrieval-Augmented Generation with Fine-Tuning}

% Anonymous submission - remove author information
% \author{Anonymous Author(s)}
% \affiliation{Anonymous Institution(s)}

\begin{document}

\maketitle

\begin{abstract}
% TODO: Write abstract (150-250 words)
% Should summarize:
% - What paper/method you reproduced
% - Key hypothesis you tested
% - Your simplified approach
% - Main findings
% - Contribution/insights

This work presents a simplified reproduction of the SimRAG (Simplified Retrieval-Augmented Generation) framework, focusing on core concepts of RAG systems and model fine-tuning. We implement a minimalist version that distills the essential components: document retrieval using semantic embeddings, vector storage with Qdrant, and fine-tuning language models for domain adaptation. Our implementation verifies the core hypothesis that combining retrieval-augmented generation with targeted fine-tuning improves question-answering performance on domain-specific documents. Through small-scale experiments with local and cloud-based LLMs, we demonstrate the feasibility of the approach and analyze the trade-offs between retrieval quality, model size, and computational resources. Our findings validate the core insights of SimRAG while providing an educational framework for understanding RAG systems.

\end{abstract}

\section{Introduction}
\label{sec:introduction}

% TODO: Expand introduction (0.5-1 page)
% Should include:
% - Motivation: Why RAG and SimRAG are important
% - Paper selection: Why you chose SimRAG
% - Problem statement: What you're trying to verify
% - Scope: What aspects you're focusing on (simplification)

\subsection{Motivation}
% TODO: Discuss importance of RAG systems in modern AI
% - How RAG addresses LLM limitations (hallucination, knowledge cutoff)
% - Why SimRAG is a good paper to reproduce
% - Educational value of understanding RAG fundamentals

\subsection{Paper Selection and Hypothesis}
% TODO: Explain why SimRAG was chosen
% - Accessible implementation requirements
% - Clear, testable claims
% - Educational value
% - State the core hypothesis you're testing:
%   "Combining retrieval-augmented generation with fine-tuning improves 
%    domain-specific question answering performance"

\subsection{Scope and Simplifications}
% TODO: Describe your simplified approach
% - What components you implemented (basic RAG, fine-tuning)
% - What you simplified or omitted (complex multi-stage training, etc.)
% - Hardware constraints (laptop vs PC configurations)
% - Justification for simplifications

\section{Related Work}
\label{sec:related}

% TODO: Write related work section (0.5 page)
% Should cover:
% - RAG systems (original RAG paper, other variants)
% - Fine-tuning approaches for domain adaptation
% - SimRAG paper overview
% - How your work relates to existing literature

\section{Method}
\label{sec:method}

% TODO: Expand method section (1-1.5 pages)
% This is the core of your reproduction study

\subsection{Hypothesis Distillation}
% TODO: Clearly state the core hypothesis
% - What claim from SimRAG are you testing?
% - How did you distill it to a testable form?
% - What are your success criteria?

\subsection{System Architecture}
% TODO: Describe your implementation architecture
% - RAG components: vector store, retriever, generator
% - Fine-tuning pipeline
% - Model versioning system
% - Configuration management

\subsubsection{Retrieval-Augmented Generation}
% TODO: Detail RAG implementation
% - Document ingestion and chunking strategy
% - Embedding model (sentence-transformers)
% - Vector store (Qdrant) setup
% - Retrieval mechanism (top-k, similarity threshold)
% - Prompt construction for generation

\subsubsection{Model Fine-Tuning}
% TODO: Describe fine-tuning approach
% - Base models used (Llama 3.2 1B, Qwen 2.5)
% - Training data preparation
% - Fine-tuning stages (if applicable)
% - Hardware-aware optimizations

\subsection{Baseline Implementation}
% TODO: Describe baseline(s) for comparison
% - Vanilla RAG (no fine-tuning)
% - Fine-tuned model without RAG
% - Original model performance
% - Justification for baseline selection

\section{Experiments}
\label{sec:experiments}

% TODO: Expand experiments section (1-1.5 pages)
% This should detail your experimental design

\subsection{Experimental Design}
% TODO: Describe experimental setup
% - Dataset: HTML documents from Docker/DevOps tutorials
% - Test questions: Domain-specific queries
% - Evaluation metrics: Context relevance scores, answer quality
% - Hardware configurations: Laptop (CPU) vs PC (GPU)

\subsection{Implementation Details}
% TODO: Technical details
% - Software stack (PyTorch, Transformers, Qdrant, etc.)
% - Model configurations
% - Training hyperparameters
% - Retrieval parameters (top-k, similarity threshold)

\subsection{Evaluation Methodology}
% TODO: How you measure success
% - Quantitative metrics: retrieval scores, response times
% - Qualitative assessment: answer relevance
% - Comparison methodology: baseline vs improved
% - Limitations of evaluation approach

\section{Results and Analysis}
\label{sec:results}

% TODO: Present and analyze results (1-1.5 pages)
% This is where you show what you found

\subsection{Retrieval Performance}
% TODO: Present retrieval results
% - Context similarity scores
% - Top-k retrieval effectiveness
% - Document relevance analysis
% - Figures/tables showing retrieval metrics

\subsection{Generation Quality}
% TODO: Analyze answer quality
% - Comparison: baseline vs fine-tuned
% - Response time analysis
% - Domain-specific improvements
% - Examples of good/poor retrievals

\subsection{Resource Analysis}
% TODO: Computational considerations
% - Training time comparisons (laptop vs PC)
% - Memory requirements
% - Model size trade-offs
% - Scalability observations

\subsection{Discussion}
% TODO: Interpret results
% - Do results support the hypothesis?
% - What worked well? What didn't?
% - Limitations and challenges encountered
% - Insights gained from reproduction

\section{Conclusion}
\label{sec:conclusion}

% TODO: Write conclusion (0.5 page)
% Should summarize:
% - Key findings
% - Verification of core hypothesis
% - Contributions of the reproduction
% - Future work or extensions
% - Lessons learned

\section*{Acknowledgments}
% TODO: Optional acknowledgments
% Keep anonymous if needed

\bibliography{references}
\bibliographystyle{plain}

% Appendices (if needed, don't count toward page limit)
\begin{appendix}
\section{Additional Results}
% TODO: Additional figures, tables, or detailed results

\section{Implementation Details}
% TODO: Code snippets, configuration examples, etc.

\end{appendix}

\end{document}

